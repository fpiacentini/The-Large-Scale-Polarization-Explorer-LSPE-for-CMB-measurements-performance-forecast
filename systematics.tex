\comment{Direi che un discorso completo sulle sistematiche richiederebbe un altro articolo. Troppi aspetti da trattare.
Qui ci possiamo limitare a commentare sulle tecniche di mitigazione e di calibrazione che intendiamo utilizzare}

\comment{Piacentini, Natoli, Bersanelli, Tomasi, Columbro, de Gasperis}

\subsection{LSPE-STRIP systematic effects and calibration}
\comment{Bersanelli, Tomasi, Mennella}


\subsection{LSPE-SWIPE systematic effects and calibration}
\comment{de Bernardis, Piacentini, Natoli, de Gasperis, Columbro, Lamagna}

\subsubsection{Instrument requirement}

\comment{From SWIPE Program requirement document. Questa parte pu\`o essere
ridotta, aggiungendo referenze a vari articoli in letteratura, inclusi Buzzelli et al., proceeding 
di Columbro 2018}

LSPE-SWIPE is designed to minimize instrumental polarization. This is the spurious signal resulting
from the measurement of unpolarized radiation. 
In the case of CMB, the amount of unpolarized radiation coming from
the sky is overwhelming with respect to the polarized signal, minimizing instrumental
polarization is the most important driver of instrument design. 
The requirement is that the
maximum acceptable level of instrumental polarization is 0.2\%. 
In this way we will get a constant
polarized signal lower than 6\,mK from the unpolarized background, while from CMB anisotropy we
will get at most 0.2\,$\mu$K spurious polarization, correlated to temperature fluctuations (even less at large
angular scales). The constant signal is treated as an offset in the data analysis, whose stability depends
on the stability of the gain of the electronics and of the responsivity of the detectors, and is not
synchronous with the observed sky. 
Instrumental polarization is reduced during system design using an
optical system close to on-axis, and avoiding mirrors in favor of lenses. The main design choice here
is to have the polarization modulator as the first optical elements in the system, thus relaxing
significantly the requirements on the following optical components.

The second parameter to be considered is cross-polarization. Cross-polarization is defined as the
response of a polarimeter to an input signal polarized in direction orthogonal to the nominal
polarimeter direction. Cross-polarization results in leakage of E-modes into B-modes. 
%Crosspol can be effectively mitigated by means of a precise calibration and data analysis, as confirmed by our
%experience in the data analysis of BOOMERanG and Planck-HFI, where levels of cross-polarization
%around 10% were managed successfully. 
Our requirement is that the maximum acceptable level of cross-polarization is below 2\%. 
This is achieved again by means of an accurate optical design.

The third parameter to be considered is the ellipticity of the main beam (detector angular response 
in the sky). Step rotation or spinning of
the Half Wave Plate allows to observe the same sky region with the same beam orientation, and
different polarimeter orientation. This strongly mitigates the ellipticity requirement, and differential 
ellipticity among different detectors. 
%Previous experience shows that an acceptable level of ellipticity is below 2\%. 
%The main way to achieve this
%is to use an on-axis system and not to use the marginal regions of the focal plane. In our baseline the
%used area of the focal plane is only the central 25\% of the area of the entrance pupil. However, the
%ellipticity in the beam is mainly dictated by the multimode propagation in the horns instead of
%aberration effects along the focal plane. So a smarter optical configuration is under study, which
%should allow to re-gain a larger fraction of the area.

Correct measurement of the angles of the polarimeters is crucial to avoid leakage from E-modes into
B-modes, and to avoid contamination in the measurement of fundamental physics effects such as
cosmic birifringence. In this contest, our system is characterized by the presence of a single, large
wire grid polarizer, defining the reference system for polarization measurements. With this design
the system is similar to an ideal polarimeter, and the angle of the single large wire grid polarimeter
can be accurately measured. We set the requirement for the polarimeter angle measurement to
0.2 degrees. This will improve by at least a factor 10 with respect to e.g. Planck-HFI, where the
polarization axis is defined by the orientation of PSBs, much more difficult to measure due to their
small dimensions \citep{rosset2010}.

Spectral matching among detectors 
has been historically a problem for instruments without a polarization modulator, just comparing
two independent measurements of the orthogonal polarization components. 
In LSPE-SWIPE we use a Stokes polarimeter, where the same detector measures both polarizations,
alternated by means of a rotating half-wave plate. In this configuration the most important requirement is
that the waveplate has high modulation efficiency over the detection bandwidth of the focal plane it
serves. In our system a single waveplate covers all the bands from 120 to 250 GHz. This
requires 70\% bandwidth for the waveplate, a goal certainly reachable with significant
accuracy, by means of metamaterials \citep[see][]{Pisano:06}.
%In addition, band-defining filters and blocking filters should not produce a significant background on
%the detectors and should not allow any leak of high frequency radiation (up to the UV band). We will
%use the same requirements as in the Planck-HFI instrument.

\subsubsection{Half-Wave Plate synchronous effects}
\comment{Columbro }

\citep[see][]{ritacco2017}

\subsubsection{LSPE-SWIPE calibration }
\comment{de Bernardis, Piacentini, Natoli, de Gasperis, Columbro, Lamagna}


