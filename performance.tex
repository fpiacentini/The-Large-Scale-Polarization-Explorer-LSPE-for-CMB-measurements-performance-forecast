\subsection{Instrument simulators}
\comment{Piacentini, Tomasi }
Simulators are key elements for instrument design and for data analysis. 
In the design phase, they allow to predict the scientific performance of the 
instruments and the impact of systematic effects. In the data analysis phase, 
they allow to run Montecarlo realizations of the observations, which are 
necessary to estimate instrumental biases, measure transfer functions, 
and to propagate uncertainties. 

The instrument simulator of LSPE-SWIPE consists in a parallel Fortran-90 code
which takes as input:
\begin{itemize} 
\item the number of detectors;
\item the detectors position in the focal plane; 
\item the detectors noise, in terms of NET and 1/f knee frequency;
\item the mission starting date and duration;
\item the angular response in the sky of each detector, as a 2d matrix; this
is convolved in pixel space, in a radius specified also as input parameter;
\item the HWP operation strategy (stepping, spinning, spinning rate, stepping period);
\item the level of HWP synchronous systematic effects, as a signal in $\mu$K$_\text{CMB}$
at the HWP spinning frequency and its harmonics;
\item HWP angle offset or angle statistical uncertainty;
\item timeline filter (high-pass, low-pass or band-pass);
\item map-making algorithm details, as simple re-binning, or iterative
destriping.
\end{itemize}
It generates in output:
\begin{itemize} 
\item timeline of each detector (optional);
\item maps of each detector;
\item map of combined detectors;
\item coverage map;
\item noise covariance 3x3 matrix for each observed pixel.
\end{itemize}

The instrument simulator of LSPE-STRIP is based on ... \comment{Tommasi}


\subsection{Noise estimation}
\comment{de Bernardis for SWIPE, including distribution of detector number in the bands}\\
\comment{Tomasi/Mennella for STRIP}


\subsection{Sky coverage}
\comment{Discuss Swipe/Strip the best observation strategy to maximise overlap.
How low can Strip go in zenithal angle to match Swipe coverage }

\subsection{Component separation}
\comment{Molinari, Pagano, Natoli, Baccigalupi, ...}
Component separation is a mandatory step in the extraction of the CMB polarization 
\citep[see e.g.][]{buzzelli_migliaccio2018}. In particular Galactic Dust and Synchrotron 
emission are the most relevant foreground in polarization. 
In this analysis we consider a template fitting technique, in which the 140 GHz channel
is used to extract CMB polarization, while other channels are used as 
foreground templates to remove foregrounds.
For example, a high frequency channel map, $M_\text{HF}$, can be used 
to trace and remove the dust signal, $M_\text{dust}$, from the 
140\,GHz map, $M_\text{140}$.   
The high frequency map contains dust and CMB signal:
\begin{equation} 
M_\text{HF} = M_\text{CMB} + M_\text{dust}
\end{equation}
while the 140 GHz channel contains CMB and a fraction $\alpha_\text{dust}$ of
the dust in the $M_\text{HF}$ map:
\begin{equation} 
M_\text{140} = M_\text{CMB} + \alpha_\text{dust} \cdot M_\text{dust}
\end{equation}
The CMB map can be extracted as
\begin{equation} 
M_\text{CMB} = \frac{M_\text{140} - \alpha_\text{dust} \cdot M_\text{HF}}{1-\alpha_\text{dust}}
\end{equation}
The noise, $N_\text{CMB}$, in the CMB reconstructed map is then
\begin{equation}
N_\text{CMB}^2 = \left( \frac{1}{1-\alpha_\text{dust}} \right)^2 N_\text{140}^2 + \left( \frac{\alpha_\text{dust}}{1-\alpha_\text{dust}} \right)^2 N_\text{HF}^2
\end{equation}
where $N_\text{140}$ is the noise in the 140\,GHz map, and $N_\text{HF}$ is the noise in the high 
frequency map. 
The coefficient that multiplies the noise in the 140\,GHz map is always larger than 1;
the coefficient that multiplies the noise in the HF map is smaller than 1 if $\alpha_\text{dust} < 0.5$, 
larger otherwise. 
In principle the $\alpha_\text{dust}$ coefficient may be different in different sky patches, but 
we assume a constant value in this analysis. 
\\
\comment{In this paper we mainly focus on noise propagation from template fitting 
technique; an independent paper can describe in detail the template fitting technique}\\
\comment{Include a figure with LSPE frequency bands and spectra of the polarised 
foregrounds } \\
\comment{Dobbiamo decidere come trattare il sincrotrone. In prima approssimazione 
possiamo scrivere che il rumore delle mappe a bassa frequenza, propagato usando 
l'indice spettrale, \`e trascurabile rispetto al rumore della banda a 140 GHz. 
Poi aggiungere un commento su possibile bias dovuto a curvatura dello spettro 
del sincrotrone e spettro non uniforme. In modo da evidenziare l'importanza 
dei canali a bassa frequenza. }

\subsection{Likelihood}
\comment{Pagano/Natoli}

\section{Performance forecast, results}\label{sec:results}

Instrumental configurations: SWIPE flight 8/15 days; SWIPE coverage; STRIP coverage.


See table~\ref{tab:tauerrors} for result on the optical depth $\tau$.\comment{Pagano} 

%
%\begin{table}[htp]
%\begin{center}
%\begin{tabular}{l|c|c|c|c}
%channels & v01 & v02 & v03 &v04 \\
%\hline
%140 & 0.0038 & 0.0043 & 0.0036 & 0.0040\\
%140 c. 220 & 0.0044 & 0.0051 & 0.0042 & 0.0049\\
%140 c. 240 & 0.0049 & 0.0060 & 0.0047 & 0.0056\\
%140 c. 255 & 0.0048 & 0.0056 & 0.0045 & 0.0056\\
%140 c. 270 & 0.0045 & 0.0052 & 0.0043 & 0.0049\\
%140 c. 353 & 0.0042 & 0.0046 & 0.0040 & 0.0043\\
%220 c. 240 & 0.0106 & 0.0138 & 0.0106 & 0.0136\\
%220 c. 255 & 0.0104 & 0.0128 & 0.0097 & 0.0123\\
%220 c. 270 & 0.0096 & 0.0116 & 0.0090 & 0.0116\\
%220 c. 353 & 0.0088 & 0.0102 & 0.0084 & 0.0102\\
%140 c. 353 + 220 c. 270  & 0.0042 & 0.0045 & 0.0039 & 0.0043\\
%140 c. 220 240 353  & 0.0040 & 0.0045 & 0.0038 & 0.0044\\
%140 c. 220 270 353  & 0.0039 & 0.0045 & 0.0039 & 0.0041\\\end{tabular}
%\end{center}
%\caption{$\tau$ 1-$\sigma$ errors obtained marginalizing over $\ln(10^{10}A_{\rm s})$. "c." means "cleaned by".}
%\label{tab:tauerrors}
%\end{table}%


\begin{table}[htp]
\begin{center}
\begin{tabular}{l|c|c}
channels & 8 days & 15 days \\
\hline
140 & 0.0037 & 0.0042\\
140 c. 220 & 0.0048 & 0.0061\\
140 c. 220x2 & 0.0045 & 0.0054\\
140 c. 240 & 0.0053 & 0.0065\\
140 c. 255 & 0.0051 & 0.0059\\
140 c. 270 & 0.0046 & 0.0054\\
140 c. 353 & 0.0041 & 0.0046\\
140 c. 240 + 220 c. 353  & 0.0051 & 0.0061\\
140 c. 255 + 220 c. 353  & 0.0047 & 0.0055\\
140 c. 270 + 220 c. 353  & 0.0046 & 0.0054\\
140 c. 353 + 220 cc. 270  & 0.0041 & 0.0044\\
\end{tabular}
\end{center}
\caption{$\tau$ 1-$\sigma$ errors obtained marginalizing over $\ln(10^{10}A_{\rm s})$. "c." means "cleaned by".
\comment{Scambiare intestazione colonne}}
\label{tab:tauerrors}
\end{table}%

See table~\ref{tab:rerrors} for uncertainty estimation on the tensor to scalar ratio $r$.\comment{Pagano} 

\begin{table}[htp]
\begin{center}
\begin{tabular}{l|c|c}
channels & 8 days & 15 days \\
\hline
140 & 0.011 & 0.019\\
140 c. 220 & 0.036 & 0.059\\
140 c. 220x2 & 0.027 & 0.046\\
140 c. 240 & 0.048 & 0.071\\
140 c. 270 & 0.028 & 0.048\\
140 c. 353 & 0.019 & 0.027\\
140 c. 353 + 220 cc. 270  & 0.019 & 0.027\\
\end{tabular}
\end{center}
\caption{Upper limit at 68\% C.L. on $r$. "c." means "cleaned by". 
\comment{Scambiare intestazione colonne. I valori devono essere aggiornati. }}
\label{tab:rerrors}
\end{table}%

~\\
Spectral index? \comment{possiamo dire qualcosa?} \\ \\
Low-ell anomaly \comment{Gruppuso/Natoli}\\ \\
Birefringence \comment{Melchiorri++} \\ \\
Other science: \\
- non-gaussianity \comment{Matarrese?} , \\
- galactic synchrotron \comment{Nicoletta}, \\
- dust \comment{Masi}

