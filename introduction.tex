\comment{From LSPE program requirement; add references}\\
The Large Scale Polarization explorer (LSPE) is designed to measure the polarization of the 
Cosmic Microwave Background (CMB) at large
angular scales, and in particular to constrain the curl component of CMB polarization (B-modes).
This is produced by tensor perturbations generated during cosmic inflation, in the very early Universe.
The level of this signal is unknown: current inflation models are unable to provide a reference
value. However, the detection of this signal would be of utmost importance, providing a way to
measure the energy-scale of inflation and a window on the physics at extremely high energies.
While the level of CMB anisotropy is of the order of $100\,\mu$K rms and the level of the gradient
component of CMB polarization (E-modes generated by scalar - density perturbations) is of the order
of $3\,\mu$K, the current upper limits for the level of B-modes polarization are a fraction of $\mu$K,
corresponding to a ratio between the amplitude of tensor perturbations and the amplitude of scalar
perturbations $r < 0.07$ (68\% CL, see~\citet{bicep2018}). 
The target of LSPE is to improve this limit.

Additional targets of the mission are 
an improved measure of the optical depth to the 
cosmic microwave background $\tau$, measured from the large scale
E-mode CMB polarization; investigation of the so called
{\em low-$\ell$ anomaly}, a series of anomalies observed in the
large angular scales of the CMB polarization, including
lack of power, asymmetries and alignment of multipole moments \citep{planckIandS2016}
\comment{Gruppuso?}; 
wide maps of foreground polarization produced in our galaxy
by synchrotron emission and interstellar dust emission, whick will be important to map the magnetic
field in our Galaxy and to study the properties of the ionized gas and of the diffuse interstellar dust
in the Milky Way.

We set the target for the LSPE program a measurement of B-modes of CMB polarization
at a level corresponding to a tensor to scalar ratio r=0.03 with 99.7\% confidence (r=0.013 at 68\% CL)
\comment{rivedere}.
Since the expected B-mode signal in this case is smaller than the polarized foreground from our
Galaxy, a wide frequency coverage is needed to monitor precisely the foregrounds at frequencies
where they are most important, and subtract them to estimate the cosmological part of the detected
B-mode signal. For the synchrotron foreground, prominent at frequencies lower than the peak
frequency of the CMB (160 GHz), where atmospheric transmission and noise are favorable, a ground
based instrument is the most effective way to go. For the interstellar dust foreground, prominent at
frequencies higher than 160GHz, where atmospheric transmission and noise are poor, a stratospheric
balloon mission is required. For this reason (as described in greater detail below), the LSPE program
is composed of two experiments: a ground-based experiment, running the STRIP instrument
observing at 44 GHz, and a balloon-borne mission, flying the SWIPE instrument observing at 140,
220, 240 GHz.

In section~\ref{sec:instruments} we describe the two instruments with some detail. 
In section~\ref{sec:methods} we present the methods used to forecast the experiment
performance. 
In section~\ref{sec:results} we report the expected results. 
In section~\ref{sec:systematics} we describe the major systematic effects, and their 
mitigation techniques. 
Finally, in section~\ref{sec:conclusion} we draw conclusions. 
